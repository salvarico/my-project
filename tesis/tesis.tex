

% Default to the notebook output style

    


% Inherit from the specified cell style.





    
\documentclass[titlepage]{article} % this was modified, the default is \documentclass{article}

    
    
    \usepackage{graphicx} % Used to insert images
    \usepackage{adjustbox} % Used to constrain images to a maximum size 
    \usepackage{color} % Allow colors to be defined
    \usepackage{enumerate} % Needed for markdown enumerations to work
    \usepackage{geometry} % Used to adjust the document margins
    \usepackage{amsmath} % Equations
    \usepackage{amssymb} % Equations
    \usepackage[mathletters]{ucs} % Extended unicode (utf-8) support
    \usepackage[utf8x]{inputenc} % Allow utf-8 characters in the tex document
    \usepackage{fancyvrb} % verbatim replacement that allows latex
    \usepackage{grffile} % extends the file name processing of package graphics 
                         % to support a larger range 
    % The hyperref package gives us a pdf with properly built
    % internal navigation ('pdf bookmarks' for the table of contents,
    % internal cross-reference links, web links for URLs, etc.)
    \usepackage{hyperref}
    \usepackage{longtable} % longtable support required by pandoc >1.10
    \usepackage{booktabs}  % table support for pandoc > 1.12.2
    

    
    
    \definecolor{orange}{cmyk}{0,0.4,0.8,0.2}
    \definecolor{darkorange}{rgb}{.71,0.21,0.01}
    \definecolor{darkgreen}{rgb}{.12,.54,.11}
    \definecolor{myteal}{rgb}{.26, .44, .56}
    \definecolor{gray}{gray}{0.45}
    \definecolor{lightgray}{gray}{.95}
    \definecolor{mediumgray}{gray}{.8}
    \definecolor{inputbackground}{rgb}{.95, .95, .85}
    \definecolor{outputbackground}{rgb}{.95, .95, .95}
    \definecolor{traceback}{rgb}{1, .95, .95}
    % ansi colors
    \definecolor{red}{rgb}{.6,0,0}
    \definecolor{green}{rgb}{0,.65,0}
    \definecolor{brown}{rgb}{0.6,0.6,0}
    \definecolor{blue}{rgb}{0,.145,.698}
    \definecolor{purple}{rgb}{.698,.145,.698}
    \definecolor{cyan}{rgb}{0,.698,.698}
    \definecolor{lightgray}{gray}{0.5}
    
    % bright ansi colors
    \definecolor{darkgray}{gray}{0.25}
    \definecolor{lightred}{rgb}{1.0,0.39,0.28}
    \definecolor{lightgreen}{rgb}{0.48,0.99,0.0}
    \definecolor{lightblue}{rgb}{0.53,0.81,0.92}
    \definecolor{lightpurple}{rgb}{0.87,0.63,0.87}
    \definecolor{lightcyan}{rgb}{0.5,1.0,0.83}
    
    % commands and environments needed by pandoc snippets
    % extracted from the output of `pandoc -s`
    \DefineVerbatimEnvironment{Highlighting}{Verbatim}{commandchars=\\\{\}}
    % Add ',fontsize=\small' for more characters per line
    \newenvironment{Shaded}{}{}
    \newcommand{\KeywordTok}[1]{\textcolor[rgb]{0.00,0.44,0.13}{\textbf{{#1}}}}
    \newcommand{\DataTypeTok}[1]{\textcolor[rgb]{0.56,0.13,0.00}{{#1}}}
    \newcommand{\DecValTok}[1]{\textcolor[rgb]{0.25,0.63,0.44}{{#1}}}
    \newcommand{\BaseNTok}[1]{\textcolor[rgb]{0.25,0.63,0.44}{{#1}}}
    \newcommand{\FloatTok}[1]{\textcolor[rgb]{0.25,0.63,0.44}{{#1}}}
    \newcommand{\CharTok}[1]{\textcolor[rgb]{0.25,0.44,0.63}{{#1}}}
    \newcommand{\StringTok}[1]{\textcolor[rgb]{0.25,0.44,0.63}{{#1}}}
    \newcommand{\CommentTok}[1]{\textcolor[rgb]{0.38,0.63,0.69}{\textit{{#1}}}}
    \newcommand{\OtherTok}[1]{\textcolor[rgb]{0.00,0.44,0.13}{{#1}}}
    \newcommand{\AlertTok}[1]{\textcolor[rgb]{1.00,0.00,0.00}{\textbf{{#1}}}}
    \newcommand{\FunctionTok}[1]{\textcolor[rgb]{0.02,0.16,0.49}{{#1}}}
    \newcommand{\RegionMarkerTok}[1]{{#1}}
    \newcommand{\ErrorTok}[1]{\textcolor[rgb]{1.00,0.00,0.00}{\textbf{{#1}}}}
    \newcommand{\NormalTok}[1]{{#1}}
    
    % Define a nice break command that doesn't care if a line doesn't already
    % exist.
    \def\br{\hspace*{\fill} \\* }
    % Math Jax compatability definitions
    \def\gt{>}
    \def\lt{<}
    % Document parameters
    
\title{Detección de animales}

    
    
\author{José Salvador Rico Mercado}

    

    % Pygments definitions
    
\makeatletter
\def\PY@reset{\let\PY@it=\relax \let\PY@bf=\relax%
    \let\PY@ul=\relax \let\PY@tc=\relax%
    \let\PY@bc=\relax \let\PY@ff=\relax}
\def\PY@tok#1{\csname PY@tok@#1\endcsname}
\def\PY@toks#1+{\ifx\relax#1\empty\else%
    \PY@tok{#1}\expandafter\PY@toks\fi}
\def\PY@do#1{\PY@bc{\PY@tc{\PY@ul{%
    \PY@it{\PY@bf{\PY@ff{#1}}}}}}}
\def\PY#1#2{\PY@reset\PY@toks#1+\relax+\PY@do{#2}}

\expandafter\def\csname PY@tok@gd\endcsname{\def\PY@tc##1{\textcolor[rgb]{0.63,0.00,0.00}{##1}}}
\expandafter\def\csname PY@tok@gu\endcsname{\let\PY@bf=\textbf\def\PY@tc##1{\textcolor[rgb]{0.50,0.00,0.50}{##1}}}
\expandafter\def\csname PY@tok@gt\endcsname{\def\PY@tc##1{\textcolor[rgb]{0.00,0.27,0.87}{##1}}}
\expandafter\def\csname PY@tok@gs\endcsname{\let\PY@bf=\textbf}
\expandafter\def\csname PY@tok@gr\endcsname{\def\PY@tc##1{\textcolor[rgb]{1.00,0.00,0.00}{##1}}}
\expandafter\def\csname PY@tok@cm\endcsname{\let\PY@it=\textit\def\PY@tc##1{\textcolor[rgb]{0.25,0.50,0.50}{##1}}}
\expandafter\def\csname PY@tok@vg\endcsname{\def\PY@tc##1{\textcolor[rgb]{0.10,0.09,0.49}{##1}}}
\expandafter\def\csname PY@tok@m\endcsname{\def\PY@tc##1{\textcolor[rgb]{0.40,0.40,0.40}{##1}}}
\expandafter\def\csname PY@tok@mh\endcsname{\def\PY@tc##1{\textcolor[rgb]{0.40,0.40,0.40}{##1}}}
\expandafter\def\csname PY@tok@go\endcsname{\def\PY@tc##1{\textcolor[rgb]{0.53,0.53,0.53}{##1}}}
\expandafter\def\csname PY@tok@ge\endcsname{\let\PY@it=\textit}
\expandafter\def\csname PY@tok@vc\endcsname{\def\PY@tc##1{\textcolor[rgb]{0.10,0.09,0.49}{##1}}}
\expandafter\def\csname PY@tok@il\endcsname{\def\PY@tc##1{\textcolor[rgb]{0.40,0.40,0.40}{##1}}}
\expandafter\def\csname PY@tok@cs\endcsname{\let\PY@it=\textit\def\PY@tc##1{\textcolor[rgb]{0.25,0.50,0.50}{##1}}}
\expandafter\def\csname PY@tok@cp\endcsname{\def\PY@tc##1{\textcolor[rgb]{0.74,0.48,0.00}{##1}}}
\expandafter\def\csname PY@tok@gi\endcsname{\def\PY@tc##1{\textcolor[rgb]{0.00,0.63,0.00}{##1}}}
\expandafter\def\csname PY@tok@gh\endcsname{\let\PY@bf=\textbf\def\PY@tc##1{\textcolor[rgb]{0.00,0.00,0.50}{##1}}}
\expandafter\def\csname PY@tok@ni\endcsname{\let\PY@bf=\textbf\def\PY@tc##1{\textcolor[rgb]{0.60,0.60,0.60}{##1}}}
\expandafter\def\csname PY@tok@nl\endcsname{\def\PY@tc##1{\textcolor[rgb]{0.63,0.63,0.00}{##1}}}
\expandafter\def\csname PY@tok@nn\endcsname{\let\PY@bf=\textbf\def\PY@tc##1{\textcolor[rgb]{0.00,0.00,1.00}{##1}}}
\expandafter\def\csname PY@tok@no\endcsname{\def\PY@tc##1{\textcolor[rgb]{0.53,0.00,0.00}{##1}}}
\expandafter\def\csname PY@tok@na\endcsname{\def\PY@tc##1{\textcolor[rgb]{0.49,0.56,0.16}{##1}}}
\expandafter\def\csname PY@tok@nb\endcsname{\def\PY@tc##1{\textcolor[rgb]{0.00,0.50,0.00}{##1}}}
\expandafter\def\csname PY@tok@nc\endcsname{\let\PY@bf=\textbf\def\PY@tc##1{\textcolor[rgb]{0.00,0.00,1.00}{##1}}}
\expandafter\def\csname PY@tok@nd\endcsname{\def\PY@tc##1{\textcolor[rgb]{0.67,0.13,1.00}{##1}}}
\expandafter\def\csname PY@tok@ne\endcsname{\let\PY@bf=\textbf\def\PY@tc##1{\textcolor[rgb]{0.82,0.25,0.23}{##1}}}
\expandafter\def\csname PY@tok@nf\endcsname{\def\PY@tc##1{\textcolor[rgb]{0.00,0.00,1.00}{##1}}}
\expandafter\def\csname PY@tok@si\endcsname{\let\PY@bf=\textbf\def\PY@tc##1{\textcolor[rgb]{0.73,0.40,0.53}{##1}}}
\expandafter\def\csname PY@tok@s2\endcsname{\def\PY@tc##1{\textcolor[rgb]{0.73,0.13,0.13}{##1}}}
\expandafter\def\csname PY@tok@vi\endcsname{\def\PY@tc##1{\textcolor[rgb]{0.10,0.09,0.49}{##1}}}
\expandafter\def\csname PY@tok@nt\endcsname{\let\PY@bf=\textbf\def\PY@tc##1{\textcolor[rgb]{0.00,0.50,0.00}{##1}}}
\expandafter\def\csname PY@tok@nv\endcsname{\def\PY@tc##1{\textcolor[rgb]{0.10,0.09,0.49}{##1}}}
\expandafter\def\csname PY@tok@s1\endcsname{\def\PY@tc##1{\textcolor[rgb]{0.73,0.13,0.13}{##1}}}
\expandafter\def\csname PY@tok@sh\endcsname{\def\PY@tc##1{\textcolor[rgb]{0.73,0.13,0.13}{##1}}}
\expandafter\def\csname PY@tok@sc\endcsname{\def\PY@tc##1{\textcolor[rgb]{0.73,0.13,0.13}{##1}}}
\expandafter\def\csname PY@tok@sx\endcsname{\def\PY@tc##1{\textcolor[rgb]{0.00,0.50,0.00}{##1}}}
\expandafter\def\csname PY@tok@bp\endcsname{\def\PY@tc##1{\textcolor[rgb]{0.00,0.50,0.00}{##1}}}
\expandafter\def\csname PY@tok@c1\endcsname{\let\PY@it=\textit\def\PY@tc##1{\textcolor[rgb]{0.25,0.50,0.50}{##1}}}
\expandafter\def\csname PY@tok@kc\endcsname{\let\PY@bf=\textbf\def\PY@tc##1{\textcolor[rgb]{0.00,0.50,0.00}{##1}}}
\expandafter\def\csname PY@tok@c\endcsname{\let\PY@it=\textit\def\PY@tc##1{\textcolor[rgb]{0.25,0.50,0.50}{##1}}}
\expandafter\def\csname PY@tok@mf\endcsname{\def\PY@tc##1{\textcolor[rgb]{0.40,0.40,0.40}{##1}}}
\expandafter\def\csname PY@tok@err\endcsname{\def\PY@bc##1{\setlength{\fboxsep}{0pt}\fcolorbox[rgb]{1.00,0.00,0.00}{1,1,1}{\strut ##1}}}
\expandafter\def\csname PY@tok@kd\endcsname{\let\PY@bf=\textbf\def\PY@tc##1{\textcolor[rgb]{0.00,0.50,0.00}{##1}}}
\expandafter\def\csname PY@tok@ss\endcsname{\def\PY@tc##1{\textcolor[rgb]{0.10,0.09,0.49}{##1}}}
\expandafter\def\csname PY@tok@sr\endcsname{\def\PY@tc##1{\textcolor[rgb]{0.73,0.40,0.53}{##1}}}
\expandafter\def\csname PY@tok@mo\endcsname{\def\PY@tc##1{\textcolor[rgb]{0.40,0.40,0.40}{##1}}}
\expandafter\def\csname PY@tok@kn\endcsname{\let\PY@bf=\textbf\def\PY@tc##1{\textcolor[rgb]{0.00,0.50,0.00}{##1}}}
\expandafter\def\csname PY@tok@mi\endcsname{\def\PY@tc##1{\textcolor[rgb]{0.40,0.40,0.40}{##1}}}
\expandafter\def\csname PY@tok@gp\endcsname{\let\PY@bf=\textbf\def\PY@tc##1{\textcolor[rgb]{0.00,0.00,0.50}{##1}}}
\expandafter\def\csname PY@tok@o\endcsname{\def\PY@tc##1{\textcolor[rgb]{0.40,0.40,0.40}{##1}}}
\expandafter\def\csname PY@tok@kr\endcsname{\let\PY@bf=\textbf\def\PY@tc##1{\textcolor[rgb]{0.00,0.50,0.00}{##1}}}
\expandafter\def\csname PY@tok@s\endcsname{\def\PY@tc##1{\textcolor[rgb]{0.73,0.13,0.13}{##1}}}
\expandafter\def\csname PY@tok@kp\endcsname{\def\PY@tc##1{\textcolor[rgb]{0.00,0.50,0.00}{##1}}}
\expandafter\def\csname PY@tok@w\endcsname{\def\PY@tc##1{\textcolor[rgb]{0.73,0.73,0.73}{##1}}}
\expandafter\def\csname PY@tok@kt\endcsname{\def\PY@tc##1{\textcolor[rgb]{0.69,0.00,0.25}{##1}}}
\expandafter\def\csname PY@tok@ow\endcsname{\let\PY@bf=\textbf\def\PY@tc##1{\textcolor[rgb]{0.67,0.13,1.00}{##1}}}
\expandafter\def\csname PY@tok@sb\endcsname{\def\PY@tc##1{\textcolor[rgb]{0.73,0.13,0.13}{##1}}}
\expandafter\def\csname PY@tok@k\endcsname{\let\PY@bf=\textbf\def\PY@tc##1{\textcolor[rgb]{0.00,0.50,0.00}{##1}}}
\expandafter\def\csname PY@tok@se\endcsname{\let\PY@bf=\textbf\def\PY@tc##1{\textcolor[rgb]{0.73,0.40,0.13}{##1}}}
\expandafter\def\csname PY@tok@sd\endcsname{\let\PY@it=\textit\def\PY@tc##1{\textcolor[rgb]{0.73,0.13,0.13}{##1}}}

\def\PYZbs{\char`\\}
\def\PYZus{\char`\_}
\def\PYZob{\char`\{}
\def\PYZcb{\char`\}}
\def\PYZca{\char`\^}
\def\PYZam{\char`\&}
\def\PYZlt{\char`\<}
\def\PYZgt{\char`\>}
\def\PYZsh{\char`\#}
\def\PYZpc{\char`\%}
\def\PYZdl{\char`\$}
\def\PYZhy{\char`\-}
\def\PYZsq{\char`\'}
\def\PYZdq{\char`\"}
\def\PYZti{\char`\~}
% for compatibility with earlier versions
\def\PYZat{@}
\def\PYZlb{[}
\def\PYZrb{]}
\makeatother


    % Exact colors from NB
    \definecolor{incolor}{rgb}{0.0, 0.0, 0.5}
    \definecolor{outcolor}{rgb}{0.545, 0.0, 0.0}



    
    % Prevent overflowing lines due to hard-to-break entities
    \sloppy 
    % Setup hyperref package
    \hypersetup{
      breaklinks=true,  % so long urls are correctly broken across lines
      colorlinks=true,
      urlcolor=blue,
      linkcolor=darkorange,
      citecolor=darkgreen,
      }
    % Slightly bigger margins than the latex defaults
    
    \geometry{verbose,tmargin=1in,bmargin=1in,lmargin=1in,rmargin=1in}
    
    

\usepackage[spanish]{babel}

    \begin{document}
    
    
    
    \maketitle
    
    
	Agradecimientos: Tere :)
	\newpage
    \tableofcontents
	\newpage


    
    \section{Bla, bla, bla}\label{bla-bla-bla}

Aquí hay algunos ejemplos de lo chido que está el notebook y cómo se ve
en el pdf. Lo chido es que ya no hay que preocuparse por el .tex ni nada
de eso. Lo único que hay que poner aquí son las citas bibliográficas
porque obviamente esas no se ponen solas jaja. Checa este libro generado
en IPython:
\href{https://camdavidsonpilon.github.io/Probabilistic-Programming-and-Bayesian-Methods-for-Hackers/}{Probabilistic
Programming and Bayesian Methods}. Está chingón!

Dame tus críticas!

    \subsection{The Philosophy of Bayesian
Inference}\label{the-philosophy-of-bayesian-inference}

\begin{quote}
You are a skilled programmer, but bugs still slip into your code. After
a particularly difficult implementation of an algorithm, you decide to
test your code on a trivial example. It passes. You test the code on a
harder problem. It passes once again. And it passes the next, \emph{even
more difficult}, test too! You are starting to believe that there may be
no bugs in this code\ldots{} \cite{4th-paradigm}
\end{quote}

    \subsubsection{The Bayesian state of
mind}\label{the-bayesian-state-of-mind}

Consider the following examples demonstrating the relationship between
individual beliefs and probabilities:

\begin{itemize}
\item
  I flip a coin, and we both guess the result. We would both agree,
  assuming the coin is fair, that the probability of Heads is 1/2.
  Assume, then, that I peek at the coin. Now I know for certain what the
  result is: I assign probability 1.0 to either Heads or Tails
  (whichever it is). Now what is \emph{your} belief that the coin is
  Heads? My knowledge of the outcome has not changed the coin's results.
  Thus we assign different probabilities to the result.
\item
  Your code either has a bug in it or not, but we do not know for
  certain which is true, though we have a belief about the presence or
  absence of a bug.
\item
  A medical patient is exhibiting symptoms $x$, $y$ and $z$. There are a
  number of diseases that could be causing all of them, but only a
  single disease is present. A doctor has beliefs about which disease,
  but a second doctor may have slightly different beliefs. \cite{Hef10}
\end{itemize}

For example, consider the posterior probabilities (read: posterior
beliefs) of the above examples, after observing some evidence $X$:

1. $P(A): \;\;$ the coin has a 50 percent chance of being Heads.
$P(A | X):\;\;$ You look at the coin, observe a Heads has landed, denote
this information $X$, and trivially assign probability 1.0 to Heads and
0.0 to Tails.

2. $P(A): \;\;$ This big, complex code likely has a bug in it.
$P(A | X): \;\;$ The code passed all $X$ tests; there still might be a
bug, but its presence is less likely now. \cite{Cou10}

3. $P(A):\;\;$ The patient could have any number of diseases.
$P(A | X):\;\;$ Performing a blood test generated evidence $X$, ruling
out some of the possible diseases from consideration.
\cite{PER-GRA:2007}

    \subsubsection{Our Bayesian framework}\label{our-bayesian-framework}

Secondly, we observe our evidence \cite{Perez2011,ganga09,SST}

    \begin{Verbatim}[commandchars=\\\{\}]
{\color{incolor}In [{\color{incolor}5}]:} \PY{l+s+sd}{\PYZdq{}\PYZdq{}\PYZdq{}}
        \PY{l+s+sd}{The book uses a custom matplotlibrc file, which provides the unique styles for}
        \PY{l+s+sd}{matplotlib plots. If executing this book, and you wish to use the book\PYZsq{}s}
        \PY{l+s+sd}{styling, provided are two options:}
        \PY{l+s+sd}{    1. Overwrite your own matplotlibrc file with the rc\PYZhy{}file provided in the}
        \PY{l+s+sd}{       book\PYZsq{}s styles/ dir. See http://matplotlib.org/users/customizing.html}
        \PY{l+s+sd}{    2. Also in the styles is  bmh\PYZus{}matplotlibrc.json file. This can be used to}
        \PY{l+s+sd}{       update the styles in only this notebook. Try running the following code:}
        
        \PY{l+s+sd}{        import json, matplotlib}
        \PY{l+s+sd}{        s = json.load( open(\PYZdq{}../styles/bmh\PYZus{}matplotlibrc.json\PYZdq{}) )}
        \PY{l+s+sd}{        matplotlib.rcParams.update(s)}
        
        \PY{l+s+sd}{\PYZdq{}\PYZdq{}\PYZdq{}}
        
        \PY{c}{\PYZsh{} The code below can be passed over, as it is currently not important, plus it}
        \PY{c}{\PYZsh{} uses advanced topics we have not covered yet. LOOK AT PICTURE, MICHAEL!}
        \PY{o}{\PYZpc{}}\PY{k}{matplotlib} \PY{n}{inline}
        \PY{k+kn}{from} \PY{n+nn}{IPython.core.pylabtools} \PY{k+kn}{import} \PY{n}{figsize}
        \PY{k+kn}{import} \PY{n+nn}{numpy} \PY{k+kn}{as} \PY{n+nn}{np}
        \PY{k+kn}{from} \PY{n+nn}{matplotlib} \PY{k+kn}{import} \PY{n}{pyplot} \PY{k}{as} \PY{n}{plt}
        \PY{n}{figsize}\PY{p}{(}\PY{l+m+mi}{11}\PY{p}{,} \PY{l+m+mi}{9}\PY{p}{)}
        
        \PY{k+kn}{import} \PY{n+nn}{scipy.stats} \PY{k+kn}{as} \PY{n+nn}{stats}
        
        \PY{n}{dist} \PY{o}{=} \PY{n}{stats}\PY{o}{.}\PY{n}{beta}
        \PY{n}{n\PYZus{}trials} \PY{o}{=} \PY{p}{[}\PY{l+m+mi}{0}\PY{p}{,} \PY{l+m+mi}{1}\PY{p}{,} \PY{l+m+mi}{2}\PY{p}{,} \PY{l+m+mi}{3}\PY{p}{,} \PY{l+m+mi}{4}\PY{p}{,} \PY{l+m+mi}{5}\PY{p}{,} \PY{l+m+mi}{8}\PY{p}{,} \PY{l+m+mi}{15}\PY{p}{,} \PY{l+m+mi}{50}\PY{p}{,} \PY{l+m+mi}{500}\PY{p}{]}
        \PY{n}{data} \PY{o}{=} \PY{n}{stats}\PY{o}{.}\PY{n}{bernoulli}\PY{o}{.}\PY{n}{rvs}\PY{p}{(}\PY{l+m+mf}{0.5}\PY{p}{,} \PY{n}{size}\PY{o}{=}\PY{n}{n\PYZus{}trials}\PY{p}{[}\PY{o}{\PYZhy{}}\PY{l+m+mi}{1}\PY{p}{]}\PY{p}{)}
        \PY{n}{x} \PY{o}{=} \PY{n}{np}\PY{o}{.}\PY{n}{linspace}\PY{p}{(}\PY{l+m+mi}{0}\PY{p}{,} \PY{l+m+mi}{1}\PY{p}{,} \PY{l+m+mi}{100}\PY{p}{)}
        
        \PY{c}{\PYZsh{} For the already prepared, I\PYZsq{}m using Binomial\PYZsq{}s conj. prior.}
        \PY{k}{for} \PY{n}{k}\PY{p}{,} \PY{n}{N} \PY{o+ow}{in} \PY{n+nb}{enumerate}\PY{p}{(}\PY{n}{n\PYZus{}trials}\PY{p}{)}\PY{p}{:}
            \PY{n}{sx} \PY{o}{=} \PY{n}{plt}\PY{o}{.}\PY{n}{subplot}\PY{p}{(}\PY{n+nb}{len}\PY{p}{(}\PY{n}{n\PYZus{}trials}\PY{p}{)} \PY{o}{/} \PY{l+m+mi}{2}\PY{p}{,} \PY{l+m+mi}{2}\PY{p}{,} \PY{n}{k} \PY{o}{+} \PY{l+m+mi}{1}\PY{p}{)}
            \PY{n}{plt}\PY{o}{.}\PY{n}{xlabel}\PY{p}{(}\PY{l+s}{\PYZdq{}}\PY{l+s}{\PYZdl{}p\PYZdl{}, probability of heads}\PY{l+s}{\PYZdq{}}\PY{p}{)} \PYZbs{}
                \PY{k}{if} \PY{n}{k} \PY{o+ow}{in} \PY{p}{[}\PY{l+m+mi}{0}\PY{p}{,} \PY{n+nb}{len}\PY{p}{(}\PY{n}{n\PYZus{}trials}\PY{p}{)} \PY{o}{\PYZhy{}} \PY{l+m+mi}{1}\PY{p}{]} \PY{k}{else} \PY{n+nb+bp}{None}
            \PY{n}{plt}\PY{o}{.}\PY{n}{setp}\PY{p}{(}\PY{n}{sx}\PY{o}{.}\PY{n}{get\PYZus{}yticklabels}\PY{p}{(}\PY{p}{)}\PY{p}{,} \PY{n}{visible}\PY{o}{=}\PY{n+nb+bp}{False}\PY{p}{)}
            \PY{n}{heads} \PY{o}{=} \PY{n}{data}\PY{p}{[}\PY{p}{:}\PY{n}{N}\PY{p}{]}\PY{o}{.}\PY{n}{sum}\PY{p}{(}\PY{p}{)}
            \PY{n}{y} \PY{o}{=} \PY{n}{dist}\PY{o}{.}\PY{n}{pdf}\PY{p}{(}\PY{n}{x}\PY{p}{,} \PY{l+m+mi}{1} \PY{o}{+} \PY{n}{heads}\PY{p}{,} \PY{l+m+mi}{1} \PY{o}{+} \PY{n}{N} \PY{o}{\PYZhy{}} \PY{n}{heads}\PY{p}{)}
            \PY{n}{plt}\PY{o}{.}\PY{n}{plot}\PY{p}{(}\PY{n}{x}\PY{p}{,} \PY{n}{y}\PY{p}{,} \PY{n}{label}\PY{o}{=}\PY{l+s}{\PYZdq{}}\PY{l+s}{observe }\PY{l+s+si}{\PYZpc{}d}\PY{l+s}{ tosses,}\PY{l+s+se}{\PYZbs{}n}\PY{l+s}{ }\PY{l+s+si}{\PYZpc{}d}\PY{l+s}{ heads}\PY{l+s}{\PYZdq{}} \PY{o}{\PYZpc{}} \PY{p}{(}\PY{n}{N}\PY{p}{,} \PY{n}{heads}\PY{p}{)}\PY{p}{)}
            \PY{n}{plt}\PY{o}{.}\PY{n}{fill\PYZus{}between}\PY{p}{(}\PY{n}{x}\PY{p}{,} \PY{l+m+mi}{0}\PY{p}{,} \PY{n}{y}\PY{p}{,} \PY{n}{color}\PY{o}{=}\PY{l+s}{\PYZdq{}}\PY{l+s}{\PYZsh{}348ABD}\PY{l+s}{\PYZdq{}}\PY{p}{,} \PY{n}{alpha}\PY{o}{=}\PY{l+m+mf}{0.4}\PY{p}{)}
            \PY{n}{plt}\PY{o}{.}\PY{n}{vlines}\PY{p}{(}\PY{l+m+mf}{0.5}\PY{p}{,} \PY{l+m+mi}{0}\PY{p}{,} \PY{l+m+mi}{4}\PY{p}{,} \PY{n}{color}\PY{o}{=}\PY{l+s}{\PYZdq{}}\PY{l+s}{k}\PY{l+s}{\PYZdq{}}\PY{p}{,} \PY{n}{linestyles}\PY{o}{=}\PY{l+s}{\PYZdq{}}\PY{l+s}{\PYZhy{}\PYZhy{}}\PY{l+s}{\PYZdq{}}\PY{p}{,} \PY{n}{lw}\PY{o}{=}\PY{l+m+mi}{1}\PY{p}{)}
        
            \PY{n}{leg} \PY{o}{=} \PY{n}{plt}\PY{o}{.}\PY{n}{legend}\PY{p}{(}\PY{p}{)}
            \PY{n}{leg}\PY{o}{.}\PY{n}{get\PYZus{}frame}\PY{p}{(}\PY{p}{)}\PY{o}{.}\PY{n}{set\PYZus{}alpha}\PY{p}{(}\PY{l+m+mf}{0.4}\PY{p}{)}
            \PY{n}{plt}\PY{o}{.}\PY{n}{autoscale}\PY{p}{(}\PY{n}{tight}\PY{o}{=}\PY{n+nb+bp}{True}\PY{p}{)}
        
        
        \PY{n}{plt}\PY{o}{.}\PY{n}{suptitle}\PY{p}{(}\PY{l+s}{\PYZdq{}}\PY{l+s}{Bayesian updating of posterior probabilities}\PY{l+s}{\PYZdq{}}\PY{p}{,}
                     \PY{n}{y}\PY{o}{=}\PY{l+m+mf}{1.02}\PY{p}{,}
                     \PY{n}{fontsize}\PY{o}{=}\PY{l+m+mi}{14}\PY{p}{)}
        
        \PY{n}{plt}\PY{o}{.}\PY{n}{tight\PYZus{}layout}\PY{p}{(}\PY{p}{)}
\end{Verbatim}

    \begin{center}
    \adjustimage{max size={0.9\linewidth}{0.9\paperheight}}{tesis_files/tesis_4_0.png}
    \end{center}
    { \hspace*{\fill} \\}
    
    \section{Bla, bla, bla}\label{bla-bla-bla}

    \subsection{No es necesario todo el
código}\label{no-es-necesario-todo-el-cuxf3digo}

    \begin{Verbatim}[commandchars=\\\{\}]
{\color{incolor}In [{\color{incolor}14}]:} \PY{c}{\PYZsh{} Habrá codigo que sea muy largo y no tan importante y lo podemos poner en el directorio}
         \PY{o}{\PYZpc{}}\PY{k}{run} \PY{n}{imprime}\PY{o}{.}\PY{n}{py}
\end{Verbatim}

    \begin{Verbatim}[commandchars=\\\{\}]
A esto me refiero con no poner todo el codigo en el documento. Este archivo puede estar en el directorio y seria totalmente reproducible el .ipynb mientras tengas el mismo directorio
    \end{Verbatim}

    \subsubsection{Se pueden importar
imágenes}\label{se-pueden-importar-imuxe1genes}

    \begin{Verbatim}[commandchars=\\\{\}]
{\color{incolor}In [{\color{incolor}3}]:} \PY{k+kn}{from} \PY{n+nn}{IPython.display} \PY{k+kn}{import} \PY{n}{Image}
        \PY{n}{Image}\PY{p}{(}\PY{l+s}{\PYZsq{}}\PY{l+s}{http://www.funnyjunksite.com/pictures/funnypics/animals/monkey/funny\PYZus{}monkey\PYZus{}picture\PYZus{}92.jpg}\PY{l+s}{\PYZsq{}}\PY{p}{,} \PY{n}{width}\PY{o}{=}\PY{l+s}{\PYZsq{}}\PY{l+s}{100}\PY{l+s}{\PYZpc{}}\PY{l+s}{\PYZsq{}}\PY{p}{)}
\end{Verbatim}
\texttt{\color{outcolor}Out[{\color{outcolor}3}]:}
    
    \begin{center}
    \adjustimage{max size={0.9\linewidth}{0.9\paperheight}}{tesis_files/tesis_9_0.jpeg}
    \end{center}
    { \hspace*{\fill} \\}
    

    \subsection{Esto de Cython está bastante chido. Con poco esfuerzo corre
mucho más
rápido.}\label{esto-de-cython-estuxe1-bastante-chido.-con-poco-esfuerzo-corre-mucho-muxe1s-ruxe1pido.}

    \begin{Verbatim}[commandchars=\\\{\}]
{\color{incolor}In [{\color{incolor}5}]:} \PY{o}{\PYZpc{}}\PY{k}{load\PYZus{}ext} \PY{n}{cythonmagic}
\end{Verbatim}

    \begin{Verbatim}[commandchars=\\\{\}]
{\color{incolor}In [{\color{incolor}6}]:} \PY{k}{def} \PY{n+nf}{f}\PY{p}{(}\PY{n}{x}\PY{p}{)}\PY{p}{:}
            \PY{k}{return} \PY{n}{x}\PY{o}{*}\PY{o}{*}\PY{l+m+mi}{2}\PY{o}{\PYZhy{}}\PY{n}{x}
        
        \PY{k}{def} \PY{n+nf}{integrate\PYZus{}f}\PY{p}{(}\PY{n}{a}\PY{p}{,} \PY{n}{b}\PY{p}{,} \PY{n}{N}\PY{p}{)}\PY{p}{:}
            \PY{n}{s} \PY{o}{=} \PY{l+m+mi}{0}\PY{p}{;} \PY{n}{dx} \PY{o}{=} \PY{p}{(}\PY{n}{b}\PY{o}{\PYZhy{}}\PY{n}{a}\PY{p}{)}\PY{o}{/}\PY{n}{N}
            \PY{k}{for} \PY{n}{i} \PY{o+ow}{in} \PY{n+nb}{range}\PY{p}{(}\PY{n}{N}\PY{p}{)}\PY{p}{:}
                \PY{n}{s} \PY{o}{+}\PY{o}{=} \PY{n}{f}\PY{p}{(}\PY{n}{a}\PY{o}{+}\PY{n}{i}\PY{o}{*}\PY{n}{dx}\PY{p}{)}
            \PY{k}{return} \PY{n}{s}\PY{o}{*}\PY{n}{dx}
\end{Verbatim}

    \begin{Verbatim}[commandchars=\\\{\}]
{\color{incolor}In [{\color{incolor}7}]:} \PY{o}{\PYZpc{}\PYZpc{}}\PY{k}{cython}
        \PY{n}{cdef} \PY{n}{double} \PY{n}{fcy}\PY{p}{(}\PY{n}{double} \PY{n}{x}\PY{p}{)} \PY{k}{except}\PY{err}{?} \PY{o}{\PYZhy{}}\PY{l+m+mi}{2}\PY{p}{:}
            \PY{k}{return} \PY{n}{x}\PY{o}{*}\PY{o}{*}\PY{l+m+mi}{2}\PY{o}{\PYZhy{}}\PY{n}{x}
        
        \PY{k}{def} \PY{n+nf}{integrate\PYZus{}fcy}\PY{p}{(}\PY{n}{double} \PY{n}{a}\PY{p}{,}\PY{n}{double} \PY{n}{b}\PY{p}{,}\PY{n+nb}{int} \PY{n}{N}\PY{p}{)}\PY{p}{:}
            \PY{n}{cdef} \PY{n+nb}{int} \PY{n}{i}
            \PY{n}{cdef} \PY{n}{double} \PY{n}{s}\PY{p}{,} \PY{n}{dx}
            \PY{n}{s} \PY{o}{=} \PY{l+m+mi}{0}\PY{p}{;} \PY{n}{dx} \PY{o}{=} \PY{p}{(}\PY{n}{b}\PY{o}{\PYZhy{}}\PY{n}{a}\PY{p}{)}\PY{o}{/}\PY{n}{N}
            \PY{k}{for} \PY{n}{i} \PY{o+ow}{in} \PY{n+nb}{range}\PY{p}{(}\PY{n}{N}\PY{p}{)}\PY{p}{:}
                \PY{n}{s} \PY{o}{+}\PY{o}{=} \PY{n}{fcy}\PY{p}{(}\PY{n}{a}\PY{o}{+}\PY{n}{i}\PY{o}{*}\PY{n}{dx}\PY{p}{)}
            \PY{k}{return} \PY{n}{s}\PY{o}{*}\PY{n}{dx}
\end{Verbatim}

    \begin{Verbatim}[commandchars=\\\{\}]
{\color{incolor}In [{\color{incolor}8}]:} \PY{o}{\PYZpc{}}\PY{k}{timeit} \PY{n}{integrate\PYZus{}f}\PY{p}{(}\PY{l+m+mi}{0}\PY{p}{,} \PY{l+m+mi}{1}\PY{p}{,} \PY{l+m+mi}{100}\PY{p}{)}
        \PY{o}{\PYZpc{}}\PY{k}{timeit} \PY{n}{integrate\PYZus{}fcy}\PY{p}{(}\PY{l+m+mi}{0}\PY{p}{,} \PY{l+m+mi}{1}\PY{p}{,} \PY{l+m+mi}{100}\PY{p}{)}
\end{Verbatim}

    \begin{Verbatim}[commandchars=\\\{\}]
10000 loops, best of 3: 44.2 µs per loop
1000000 loops, best of 3: 747 ns per loop
    \end{Verbatim}


    % Add a bibliography block to the postdoc
    
    
\newpage
\renewcommand\refname{Bibliografía}
\bibliographystyle{unsrt}
\bibliography{ipython}

    
    \end{document}
